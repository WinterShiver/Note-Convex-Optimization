\documentclass[11pt]{ctexart}
\usepackage{titlesec}
% \titleformat{\paragraph}[block]{\normalsize\bfseries}{\theparagraph}{1em}{} % paragraph newline
\usepackage{amsmath, amsfonts, amssymb} % math

\title{\Large \textbf{优化方法作业}}
\author{\textbf{计试61 张翀 2140506063}}
\date{Week 1}

\begin{document}
\maketitle

\section*{9.4 周二}

\subsection*{作业1}

证明:
$\forall x_1, x_2\in P, \theta\in[0, 1]$,记$x_{\theta}=\theta x_1+(1-\theta)x_2$,因为$x_1, x_2\in \mathbb{R}^n$,有$x_{\theta}\in \mathbb{R}^n$.

此时$$Ax_{\theta}=A(\theta x_1+(1-\theta)x_2)=\theta Ax_1+(1-\theta)Ax_2\leq\theta b+(1-\theta)b=b$$

且有$$Bx_{\theta}=B(\theta x_1+(1-\theta)x_2)=\theta Bx_1+(1-\theta)Bx_2=\theta d+(1-\theta)d=d$$

因此$x_{\theta}\in P$.所以$P$为凸集。


\subsection*{作业2-1}

证明:
$\forall x, y\in \mathbb{R}^n, \theta\in[0, 1]$,记$x_{\theta}=\theta x+(1-\theta)y$.

此时
\begin{align*}
f(x_{\theta}) & =\underset{1\leq k\leq n}{max}(\theta x_k+(1-\theta)y_k) \\
 & =\underset{1\leq k\leq n}{sup}(\theta x_k+(1-\theta)y_k) \\
 & \leq \underset{1\leq i\leq n}{sup}(\theta x_i)+\underset{1\leq j\leq n}{sup}((1-\theta)y_j) \\
 & =\theta \underset{1\leq i\leq n}{max}(x_i)+(1-\theta)\underset{1\leq j\leq n}{max}(y_j) \\
 & =\theta f(x)+(1-\theta)f(y)
\end{align*}

$f(x)$符合凸函数的定义。

\subsection*{作业2-2}

证明:当且仅当$P$为半正定矩阵时,$f(x)=\frac{1}{2}x^TPx+q^Tx+r$为凸函数;$P$为正定矩阵时,$f(x)$为严格凸函数。此处只给出第一条结论的证明,第二条结论的证明是类似的。

固定$x$,并记$\Delta f(\delta x)=f(x+\delta x)-f(x)$;
根据定义,$f(x)$为凸函数当且仅当$\forall\theta\in[0, 1],\delta x,\theta\delta x\in\{\delta x:x+\delta x\in dom x\},\theta\Delta f(\delta x)-\Delta(\theta\delta x)\geq0$.

考虑到
\begin{align*}
\Delta(\delta x) & =f(x+\delta x)-f(x) \\
 & =(\frac{1}{2}{(x+\delta x)}^TP(x+\delta x)+q^T(x+\delta x)+r)-(\frac{1}{2}x^TPx+q^Tx+r) \\
 & =\frac{1}{2}({\delta x}^TPx+x^TP\delta x+{\delta x}^TP\delta x)+q^T\delta x \\
 & =\frac{1}{2}{\delta x}^TP\delta x+\frac{1}{2}({\delta x}^TPx+x^TP\delta x)+q^T\delta x
\end{align*}

有
$$\theta\Delta f(\delta x)-\Delta(\theta\delta x)=\frac{1}{2}\theta(1-\theta){\delta x}^TP\delta x$$

因为$\theta\in[0, 1]$,所以$$\theta\Delta f(\delta x)-\Delta(\theta\delta x)\geq0\Leftrightarrow{\delta x}^TP\delta x\geq 0, \forall\delta x\in\{\delta x:x+\delta x\in dom x\}$$

考虑到$dom x=\mathbb{R}^n$,所以$f(x)$为凸函数等价于${\delta x}^TP\delta x\geq 0, \forall\delta x\in\mathbb{R}^n$.右侧的叙述即为$P$为半正定矩阵的定义。

类似地,也可证明$P$为半负定时二次函数为凹函数,$P$为负定时二次函数严格凹。


\section*{9.6 周四}

\subsection*{作业1}

证明:二阶可微的$f(x)$为凸函数,当且仅当Hessian矩阵为半正定矩阵。

在以下的证明过程中,我们讨论一种比较简单的情况:$f(x)$在$C$上二阶连续可微,其中$C$是开集(任一元素为内点)且为凸集。此时,$\forall x\in C,\exists\delta x>0,B(x,\delta x)\subset C$,且有
$$f(x+\Delta x)=f(x)+\nabla^Tf(x)\Delta x+\frac{1}{2}\Delta x^T\nabla^2f(\xi)\Delta x,0\leq|\xi-x|\leq\Delta x\leq\delta x (*)$$

\textbf{必要性}:

若$f(x)$为凸函数,则固定$x$和$\delta x$.$\forall|\Delta x|<\delta x$,利用(*)式有
$$\theta f(x+\Delta x)+(1-\theta)f(x)-f(x+\theta\Delta x)=\frac{\theta}{2}\Delta x^T\nabla^2f(\xi)\Delta x-\frac{\theta^2}{2}\Delta x^T\nabla^2f(\xi_{\theta})\Delta x\geq0$$
$$\forall\theta\in[0,1],\xi\in B(x,\Delta x),\xi_{\theta}\in B(x,\theta\Delta x)$$

取$\theta=1$可知$\Delta x^T\nabla^2f(\xi)\Delta x-\Delta x^T\nabla^2f(\xi_{\theta})\Delta x\geq0$,而$\underset{\Delta x\to0}{lim}\Delta x^T\nabla^2f(\xi)\Delta x-\Delta x^T\nabla^2f(\xi_{\theta})\Delta x=0$.因此,固定$\theta$并取$(\frac{\theta}{2}-\frac{\theta^2}{2})\Delta x^T\nabla^2f(\xi_{\theta})\Delta x$,此式恒取非负值;因为$\underset{\Delta x\to0}{lim}\Delta x^T\nabla^2f(\xi_{\theta})\Delta x=\Delta x^T\nabla^2f(x)\Delta x$,所以由$\theta$的任意性知$\exists B(0,\delta x),\forall|\Delta x|\leq\delta x,\Delta x^T\nabla^2f(x)\Delta x\geq0$,即每个点都存在使Hessian矩阵半正定的邻域。由$C$是开集知Hessian矩阵在$C$上是半正定的。


\textbf{充分性}:

Hessian的半正定可以保证(*)式右侧最后一项非负,因此$f(x+\Delta x)\geq f(x)+\nabla^Tf(x)\Delta x$,满足凸函数的一阶判定条件,所以$f(x)$为凸函数。

\subsection*{作业2}

通过计算特征值易知$P$的正定性,所以优化函数是凸函数,且在可行解集连续可导。另外有$\nabla f(x)=Px+q$.
对于最优解$x^*=[1, 1/2, -1]^T$,$\nabla f(x^*)=[-1, 0, 2]^T$,有$\forall y\in [-1,1]^3,\nabla^Tf(x^*)(y-x^*)=y_1+2y_3+3\geq0$,因此$f(y)\geq f(x),y\in [-1,1]^3$,所以$x^*$是此问题的最优解。





\end{document}