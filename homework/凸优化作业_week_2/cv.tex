\documentclass[UTF8]{ctexart}

\usepackage{amsmath, amsfonts, amssymb}
\usepackage{geometry}
\usepackage{amsthm}
\usepackage{graphicx}
\usepackage{mathrsfs}

\usepackage{titlesec}
% \titleformat{\paragraph}[block]{\normalsize\bfseries}{\theparagraph}{1em}{} % paragraph newline

\newtheorem{thm}{定理}
\newtheorem{prop}[thm]{命题}
\newtheorem{lem}[thm]{引理}
\newtheorem{cor}{推论}
\renewcommand\proofname{证明}
 \newtheorem{pf}{证明}


\theoremstyle{Defination}
\newtheorem{defn}{定义}
\newtheorem{conj}{猜想}
\newtheorem{exmp}{例}
\theoremstyle{remark}
\newtheorem{rem}{注}


\title{\Large \textbf{优化方法作业}}
\author{\textbf{计试61 张翀 2140506063}}
\date{Week 2}
\bibliographystyle{plain}

\begin{document}
\maketitle



\section*{9.11 周二}

\subsection*{作业1-1}

\begin{lem}[上确界函数凸性质的转述]
对于函数$f(x_1,\dots,x_m,x_{m+1},\dots,x_n)$,记$$dom f(x_1,x_2,\dots,x_m)=\{(x_{m+1},\dots,x_n)|(x_1,\dots,x_m,x_{m+1},\dots,x_n)\in dom f\}$$
在$dom f$中固定$x_1,x_2,\dots,x_m$,并定义函数$g(x_{m+1},\dots,x_n)=f(x_1,\dots,x_m,x_{m+1},\dots,x_n)$.
如果$\forall x_1,x_2,\dots,x_m$,$g(x_{m+1},\dots,x_n)$是凸函数,则
$$h(x_{m+1},\dots,x_n)=\underset{x_1,\dots,x_m}{sup}{g(x_{m+1},\dots,x_n)}$$
也是凸函数。
\end{lem}

\begin{rem}
引理1转述了对讲义上关于上确界函数凸性质的定理,两者具有相同的本质。此处提出引理1以铺垫下面的证明。
\end{rem}

\begin{proof}
将$x,\lambda,v$均视为成组的变量,此时只需要证明$$-L_D(\lambda,v)=\underset{x\in dom}{sup}{-f(x)-\lambda^Tg(x)-v^Th(x)}$$对$\lambda,v$是凸函数。考虑对称性,以下只给出$-L_D(x)$对$\lambda$是凸函数的证明;$-L_D(x)$对$v$是凸函数可以类似地证明。

根据引理1,只需证明函数$-f(x)-\lambda^Tg(x)-v^Th(x)$对$\lambda$是凸函数。固定$x,v$,可知项$-f(x),-v^Th(x)$与$\lambda$的变化无关,而项$-\lambda^Tg(x)$中的$g(x)$是常向量。因此,项$-f(x),-v^Th(x)$是关于$\lambda$的常函数,而项$-\lambda^Tg(x)$是关于$\lambda$的仿射函数,它们都是关于$\lambda$的凸函数,所以其和也是关于$\lambda$的凸函数。结合引理1,可以证明$-L_D(\lambda,v)$对$\lambda$是凸函数。

考虑对称性,对$-L_D(x)$对$v$是凸函数的证明是类似的。
\end{proof}


\subsection*{作业1-2}

\begin{proof}
记问题的可行解集为$X$.

$\forall\lambda,v>0,x\in X,\lambda^Tg(x)\leq0,v^Th(x)=0$,所以
\begin{align*}
f^* & =f(x^*) \\
 & \geq f(x^*)+\lambda^Tg(x^*)+v^Th(x^*) \\
 & =L(x^*,\lambda,v) \\
 & \geq\underset{x\in X}{inf}{L(x,\lambda,v)} \\
 & =L_D(\lambda,v)
\end{align*}
\end{proof}


% \section*{9.13 周四}

% \subsection*{作业1}

\end{document}